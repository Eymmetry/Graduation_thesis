\documentclass[dvipdfmx]{jsarticle}
\usepackage{amsmath,amssymb}
%¥usepackage{mathpazo}
\usepackage{amsthm}
\newtheorem{dfn}{定義}
\newtheorem{thm}[dfn]{定理}
\newtheorem{lem}[dfn]{補題}
%\usepackage{newtxtext,newtxmath}
%\usepackage[hiresbb]{graphicx}
%¥図
\usepackage{bm}
%¥ボールド体
\usepackage{slashed}
\setlength{\textwidth}{\fullwidth}
\setlength{\textheight}{40\baselineskip}
\addtolength{\textheight}{\topskip}
\setlength{\voffset}{-0.2in}
\setlength{\topmargin}{-5pt}
\setlength{\headheight}{0pt}
\setlength{\headsep}{0pt}
\setlength{\textheight}{23.5cm}
\addtolength{\footskip}{3mm}
%\usepackage{mathrsfs}%ラグラジアン
\usepackage{wrapfig}%¥図の周りに文章を回り込ませる
\usepackage[dvipdfmx]{graphicx,color}
\usepackage{pgfplots}
\pgfplotsset{compat=1.16}
\usetikzlibrary{intersections,calc,arrows.meta,decorations.pathmorphing,backgrounds,positioning,fit,petri}
\usepackage{array}
\usepackage{siunitx}
\usepackage[version=3]{mhchem}
\usepackage{float}
\newcommand{\setN}{\mathbb{N}}
\newcommand{\setZ}{\mathbb{Z}}
\newcommand{\setQ}{\mathbb{Q}}
\newcommand{\setR}{\mathbb{R}}
\newcommand{\setC}{\mathbb{C}}
\usepackage{physics}
\usepackage{tcolorbox}
\usepackage[dvipdfmx]{hyperref}
\usepackage{pxjahyper}
\usepackage{tikz}
%\usepackage{gnuplot-lua-tikz}
\usepackage{subfigure}
\hypersetup{colorlinks=true}
\makeatletter
\@addtoreset{equation}{section}
\def\theequation{\thesection.\arabic{equation}}
\makeatother
\begin{document}
\begin{tikzpicture}
[
scale=1.0,
>=latex,
inner sep=0pt, outer sep=2pt,
temple/.style={color=black,opacity=1}
]

\filldraw[fill=gray!50!blue!10,draw=black,opacity=0.6] (1,2) rectangle (14,9) ;
\draw[thick,fill=gray!30!blue!30] (1,2)--++(.25,-0.5)--(13.75,1.5)--(14,2)--(1,2)--cycle;
\filldraw[fill=lightgray,draw=black!50] (5.5,3.5) rectangle (9.5,7.5) ;
\filldraw[fill=gray!40!black!60,draw=black!50] (5.5,3.5)--++(.10,-.25)--++(3.8,0)--++(.10,.25)--++(-4,0)--cycle;
\filldraw[fill=yellow,opacity=0.8,draw=gray!80] (1.25,2.5)--(4.25,2.5)--(4.25,7.75)--++(-.25,0)--++(0,-5)--++(-.75,0)--++(0,5)--++(-.25,0)--++(0,-5)--++(-.75,0)--++(0,5)--++(-.25,0)--++(0,-5)--++(-.75,0)--++(0,-.25)--cycle;
\filldraw[fill=orange!50!yellow!40!lightgray,opacity=0.8,draw=gray!80] (1.25,2.5)--++(3,0)--++(-.1,-.25)--++(-2.8,0)--++(-.1,.25)--cycle;


\filldraw[fill=yellow,opacity=0.8,draw=gray!80] (1.25,8.5)--++(3.5,0)--++(0,-5.5)--++(-.25,0)--++(0,5)--++(-.75,0)--++(0,-5)--++(-.25,0)--++(0,5)--++(-.75,0)--++(0,-5)--++(-.25,0)--++(0,5)--++(-1.25,0)--++(0,.5)--cycle;

\filldraw[fill=yellow, opacity=0.8,opacity=0.8,draw=gray!80] (13.75,2.5)--++(-3,0)--++(0,5.25)--++(.25,0)--++(0,-5)--++(.75,0)--++(0,5)--++(.25,0)--++(0,-5)--++(.75,0)--++(0,5)--++(.25,0)--++(0,-5)--++(.75,0)--++(0,-.25);

\filldraw[fill=yellow,opacity=0.8,draw=gray!80] (13.75,8.5)--++(-3.5,0)--++(0,-5.5)--++(.25,0)--++(0,5)--++(.75,0)--++(0,-5)--++(.25,0)--++(0,5)--++(.75,0)--++(0,-5)--++(.25,0)--++(0,5)--++(1.25,0)--++(0,.5)--cycle;

\filldraw[fill=orange!50!yellow!40!lightgray,opacity=0.8,opacity=0.8,draw=gray!80] (10.75,2.5)--++(3,0)--++(-.1,-.25)--++(-2.8,0)--++(-.1,.25)--cycle;


\node(I1) at (3,8.25){IDT1};
\node(I2) at (12,8.25){IDT2};

\draw[fill=gray,opacity=0.7] (5,9.5) rectangle (10,11.5) ;
\node(VNA) at (7.5,10.5){VNA};
\draw[thick] (5,10.5)--++(-3.5,0)--++(0,-2.25);
\draw[thick] (10,10.5)--++(3.5,0)--++(0,-2.25);
\draw[thick] (1.5,2.625)--++(0,-1.375);
\draw[thick] (1,1.25)--++(1,0);
\draw[thick] (1.125,1.125)--++(0.75,0);
\draw[thick] (1.125,1)--++(.75,0);
\draw[thick] (13.5,2.625)--++(0,-1.375);
\draw[thick] (1+12,1.25)--++(1,0);
\draw[thick] (1.125+12,1.125)--++(0.75,0);
\draw[thick] (1.125+12,1)--++(.75,0);

\draw[blue,->] (7.5-2.4,5.5)sin++(.6,.5)cos++(.6,-.5)sin++(.6,-.5)cos++(.6,.5)sin++(.6,.5)cos++(.6,-.5)sin++(.6,-.5)cos++(.6,.5);
\node[blue](RSAW) at (7.5,4.5){R-SAW};
\node(L) at (7.5,4.00){FM,NM};
\draw[<->] (12.75,3)--(12.75,7.75)node[right, midway]{ \SI{330}{\micro \metre}}  ;
\draw[<->] (10.25,5)--++(1,0)node[above, midway]{$\lambda$}  ;
\draw[<->] (4.75,8)--++(5.5,0)node[above, midway]{\SI{460}{\micro \metre}}  ;

\draw[<->] (5.75,3.5)--(5.75,7.5)node[left, midway]{ \SI{400}{\micro \metre}}  ;
\draw[<->] (5.5,7.25)--(9.5,7.25) node [below, midway]{\SI{400}{\micro \metre}};
\node(L) at (7.5,2.5){\ce{LiNbO3}};

\draw[thick,red,->] (7.5-2.4,6)--(10,6) node[above,midway]{$H$};















\end{tikzpicture}
\end{document}
