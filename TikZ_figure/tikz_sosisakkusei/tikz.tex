\documentclass[dvipdfmx]{jsarticle}
\usepackage{amsmath,amssymb}
%¥usepackage{mathpazo}
\usepackage{amsthm}
\newtheorem{dfn}{定義}
\newtheorem{thm}[dfn]{定理}
\newtheorem{lem}[dfn]{補題}
%\usepackage{newtxtext,newtxmath}
%\usepackage[hiresbb]{graphicx}
%¥図
\usepackage{bm}
%¥ボールド体
\usepackage{slashed}
\setlength{\textwidth}{\fullwidth}
\setlength{\textheight}{40\baselineskip}
\addtolength{\textheight}{\topskip}
\setlength{\voffset}{-0.2in}
\setlength{\topmargin}{-5pt}
\setlength{\headheight}{0pt}
\setlength{\headsep}{0pt}
\setlength{\textheight}{23.5cm}
\addtolength{\footskip}{3mm}
%\usepackage{mathrsfs}%ラグラジアン
\usepackage{wrapfig}%¥図の周りに文章を回り込ませる
\usepackage[dvipdfmx]{graphicx,color}
\usepackage{pgfplots}
\pgfplotsset{compat=1.16}
\usetikzlibrary{intersections,calc,arrows.meta,decorations.pathmorphing,backgrounds,positioning,fit,petri}
\usepackage{array}
\usepackage{siunitx}
\usepackage[version=3]{mhchem}
\usepackage{float}
\newcommand{\setN}{\mathbb{N}}
\newcommand{\setZ}{\mathbb{Z}}
\newcommand{\setQ}{\mathbb{Q}}
\newcommand{\setR}{\mathbb{R}}
\newcommand{\setC}{\mathbb{C}}
\usepackage{physics}
\usepackage{tcolorbox}
\usepackage[dvipdfmx]{hyperref}
\usepackage{pxjahyper}
\usepackage{tikz}
%\usepackage{gnuplot-lua-tikz}
\usepackage{subfigure}
\hypersetup{colorlinks=true}
\makeatletter
\@addtoreset{equation}{section}
\def\theequation{\thesection.\arabic{equation}}
\makeatother
\begin{document}

\begin{tikzpicture}
[
scale=1.0,
>=latex,
inner sep=0pt, outer sep=2pt,
temple/.style={color=black,opacity=1}
]
%\draw[help lines,step=0.25] (0,0) grid (16,9);
%\foreach \x in {4,8,12}
%{
%\draw[red] (\x,0)--(\x,9);
%}
%\foreach \x in {3,6}
%{
%\draw[blue] (0,\x)--(16,\x);
%}
\foreach \x / \y [count=\i] in {0/9,4/9,8/9,12/9,0/6,4/6,8/6,12/6,0/3,4/3,8/3}
{
	\coordinate(p\i)at(\x,\y);
	\node at ($(p\i)+(.25,-.25)$) {\i};
}
\foreach \i in {1,2,...,11}
{
	\draw[fill=gray!50!blue!10,draw=black] ($(p\i)+(.25,-2.75)$) rectangle ++(3.5,.5) ;
}
\foreach \i in {2,3}
{
	\draw[fill=blue,draw=black] ($(p\i)+(.25,-2.25)$) rectangle ++(3.5,.5) ;
}

\foreach \i in {7,8}
{
	\draw[fill=pink] ($(p\i)+(.25,-2.25)$) rectangle ++(3.5,.75) ;
	
}

\foreach \i in {9,10}
{
	\draw[fill=pink] ($(p\i)+(.25,-2.25)$) rectangle ++(1.25,.75) ;
	\draw[fill=pink] ($(p\i)+(2.5,-2.25)$) rectangle ++(1.25,.75) ;
	
}

\foreach \i in {10}
{
	\draw[fill=gray] ($(p\i)+(.25,-1.5)$) rectangle ++(1.25,.5) ;
	\draw[fill=gray] ($(p\i)+(2.5,-1.5)$) rectangle ++(1.25,.5) ;
}

\foreach \i in {4,5}
{
	\draw[fill=blue] ($(p\i)+(.25,-2.25)$) rectangle ++(.25,.5) ;
	\draw[fill=blue] ($(p\i)+(3.5,-2.25)$) rectangle ++(.25,.5) ;
	\draw[fill=blue] ($(p\i)+(.75,-2.25)$) rectangle ++(.25,.5) ;
	\draw[fill=blue] ($(p\i)+(3,-2.25)$) rectangle ++(.25,.5) ;
	\draw[fill=blue] ($(p\i)+(1.25,-2.25)$) rectangle ++(1.5,.5) ;
}

\foreach \i in {5}
{
	\draw[fill=yellow] ($(p\i)+(.25,-1.75)$) rectangle ++(.25,.25) ;
	\draw[fill=yellow] ($(p\i)+(3.5,-1.75)$) rectangle ++(.25,.25) ;
	\draw[fill=yellow] ($(p\i)+(.75,-1.75)$) rectangle ++(.25,.25) ;
	\draw[fill=yellow] ($(p\i)+(3,-1.75)$) rectangle ++(.25,.25) ;
	\draw[fill=yellow] ($(p\i)+(1.25,-1.75)$) rectangle ++(1.5,.25) ;
}

\foreach \i in {5,6,...,11}
{
	\draw[fill=yellow] ($(p\i)+(.5,-2.25)$) rectangle ++(.25,.25) ;
	\draw[fill=yellow] ($(p\i)+(1.,-2.25)$) rectangle ++(.25,.25) ;
	\draw[fill=yellow] ($(p\i)+(2.75,-2.25)$) rectangle ++(.25,.25) ;
	\draw[fill=yellow] ($(p\i)+(3.25,-2.25)$) rectangle ++(.25,.25) ;
}

\foreach \i in {10,11}
{
	\draw[fill=gray] ($(p\i)+(1.5,-2.25)$) rectangle ++(1,.5) ;
}

\foreach \i in {3}
{
	\draw[->] ($(p\i)+(.625,-1)$)--++(0,-.75);
	\draw[->] ($(p\i)+(.625,-1)+(.5,0)$)--++(0,-.75);
	\draw[->] ($(p\i)+(.625,-1)+(2.25,0)$)--++(0,-.75);
	\draw[->] ($(p\i)+(.625,-1)+(2.75,0)$)--++(0,-.75);
	\node at ($(p\i)+(2,-1.375)$){電子線};
}
\foreach \i in {8}
{
	\draw[->] ($(p\i)+(2,-1)$)--++(0,-.5);
	\draw[->] ($(p\i)+(2.25,-1)$)--++(0,-.5);
	\draw[->] ($(p\i)+(1.75,-1)$)--++(0,-.5);
	\node at ($(p\i)+(2,-1.375)+(1.2,.25)$){レーザー光};
}


\node at ($(p1)+(.5,-.5)$) [right] {洗浄};
\node at ($(p2)+(.5,-.5)$) [right] {レジスト塗布};
\node at ($(p3)+(.5,-.5)$) [right] {電子線描画};
\node at ($(p4)+(.5,-.5)$) [right] {現像};
\node at ($(p5)+(.5,-.5)$) [right] {Ti/Au蒸着};
\node at ($(p6)+(.5,-.5)$) [right] {リフトオフ};
\node at ($(p7)+(.5,-.5)$) [right] {レジスト塗布};
\node at ($(p8)+(.5,-.5)$) [right] {レーザー描画};
\node at ($(p9)+(.5,-.5)$) [right] {現像};
\node at ($(p10)+(.5,-.5)$)[right] {FM,NM蒸着};
\node at ($(p11)+(.5,-.5)$)[right] {リフトオフ};

\end{tikzpicture}


\end{document}
