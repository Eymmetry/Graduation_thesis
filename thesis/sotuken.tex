\RequirePackage{plautopatch}
\documentclass[dvipdfmx]{jsreport}
\usepackage{amsmath,amssymb,mathrsfs}
\usepackage{color}
\usepackage{bm}
\usepackage{cite}
\usepackage{hyperref}
\usepackage{float}
\usepackage{graphicx}
\usepackage{url}
\usepackage{braket}
\usepackage{mathtools}
\numberwithin{equation}{chapter}
\numberwithin{table}{chapter}

\begin{document}
\begin{titlepage}
	\begin{center}
		
		{\large 令和4年度}
		
		\vspace{10truept}
		
		{\large 卒業論文}
		
		\vspace*{100truept}
		
		{\huge 表面弾性波-スピン渦度結合における\\スピン軌道相互作用の寄与} 
		
		\vspace{80truept}
		
		{\LARGE 武藤永治}
		
		\vspace{5truept}
		
		{\Large 学籍番号 : 61819045}
		
		\vspace{70truept}
		
		{\Large 指導教員 : 能崎幸雄}
		
		\vspace{70truept}
		
		{\Large 慶應義塾大学}
		
		\vspace{10truept}
		
		{\Large 理工学部物理学科}
		
%		\vspace{30truept}
    
		
	\end{center}
\end{titlepage}

\setcounter{tocdepth}{3}
\tableofcontents
\clearpage

\chapter{序論}
\section{研究背景}
\subsection{スピントロニクス}
\subsection{スピン渦度結合}
\subsection{スピン渦度結合によるスピン流生成}
\section{研究目的}
\subsection{スピン軌道相互作用の寄与の解明}
\chapter{原理}
\section{スピン流}
\section{スピン渦度結合}
\section{Rayleigh表面弾性波}
\section{磁気共鳴}
\subsection{LLG方程式}
\subsection{強磁性共鳴}
\subsection{スピン波共鳴}
\chapter{実験方法}
\section{スピン流の検出手法}
\section{測定系}
\section{材料}
\chapter{実験結果}
\chapter{考察}
\chapter{まとめ}
\chapter{謝辞}
\end{document}
