\documentclass[dvipdfmx]{jsarticle}
\usepackage{amsmath,amssymb}
%¥usepackage{mathpazo}
\usepackage{amsthm}
\newtheorem{dfn}{定義}
\newtheorem{thm}[dfn]{定理}
\newtheorem{lem}[dfn]{補題}
%\usepackage{newtxtext,newtxmath}
%\usepackage[hiresbb]{graphicx}
%¥図
\usepackage{bm}
%¥ボールド体
\usepackage{slashed}
\setlength{\textwidth}{\fullwidth}
\setlength{\textheight}{40\baselineskip}
\addtolength{\textheight}{\topskip}
\setlength{\voffset}{-0.2in}
\setlength{\topmargin}{-5pt}
\setlength{\headheight}{0pt}
\setlength{\headsep}{0pt}
\setlength{\textheight}{23.5cm}
\addtolength{\footskip}{3mm}
%\usepackage{mathrsfs}%ラグラジアン
\usepackage{wrapfig}%¥図の周りに文章を回り込ませる
\usepackage[dvipdfmx]{graphicx,color}
\usepackage{pgfplots}
\pgfplotsset{compat=1.16}
\usetikzlibrary{intersections,calc,arrows,decorations.pathmorphing,backgrounds,positioning,fit,petri}
\usepackage{array}
\usepackage{siunitx}
\usepackage[version=3]{mhchem}
\usepackage{float}
\newcommand{\setN}{\mathbb{N}}
\newcommand{\setZ}{\mathbb{Z}}
\newcommand{\setQ}{\mathbb{Q}}
\newcommand{\setR}{\mathbb{R}}
\newcommand{\setC}{\mathbb{C}}
\usepackage{physics}
\usepackage{tcolorbox}
\usepackage[dvipdfmx]{hyperref}
\usepackage{pxjahyper}
\usepackage{tikz}
%\usepackage{gnuplot-lua-tikz}
\usepackage{subfigure}
\hypersetup{colorlinks=true}
\makeatletter
\@addtoreset{equation}{section}
\def\theequation{\thesection.\arabic{equation}}
\makeatother
\begin{document}
\begin{tikzpicture}
[
scale=1.0,
>=latex,
inner sep=0pt, outer sep=2pt,
temple/.style={color=black,opacity=1},
]
%\draw[help lines] (0,0) grid (10,10);
\tikzset{isometricYXZ/.style={y={(1cm,0cm)}, x={(-1.299cm,-0.75cm)}, z={(0cm,1cm)}}}

  \def \radi{3}
  \def \x{2}
  \def \y{2}
  \def \z{2}

  \begin{scope}[isometricYXZ]
   % the grid
%   \begin{scope}[color=gray!50, thin]
%    \foreach \xi in {0,...,\radi}{ \draw (\xi,\radi,0) -- (\xi,0,0) -- (\xi,0,\radi); }%
%    \foreach \yi in {1,...,\radi}{ \draw (0,\yi,\radi) -- (0,\yi,0) -- (\radi,\yi,0); }%
%    \foreach \zi in {0,...,\radi}{ \draw (0,\radi,\zi) -- (0,0,\zi) -- (\radi,0,\zi); }%
%   \end{scope}
%
   \draw[-latex, ultra thick] (0,0,0) -- (3,0,0) node[anchor=north] {X};%
   \draw[-latex, ultra thick] (0,0,0) -- (0,4,0) node[anchor=north] {Y};%
   \draw[-latex, ultra thick] (0,0,0) -- (0,0,4) node[anchor=east] {Z};%

  % \fill[color=magenta, opacity=0.2] (0,0,0) -- (\x,0,0) -- (\x,\y,0) -- (0,\y,0) -- cycle; %
  % \fill[color=yellow, opacity=0.2] (0,0,0) -- (0,0,\z) -- (0,\y,\z) -- (0,\y,0) -- cycle; %
  % \fill[color=cyan, opacity=0.2] (0,0,0) -- (\x,0,0) -- (\x,0,\z) -- (0,0,\z) -- cycle;

   \draw[color=black, thick]%
   (0,\y,\z) -- (\x,\y,\z) -- (\x,\y,0) (\x,\y,\z) -- (\x,0,\z) 
   (\x,0,\z)--(0,0,\z)--(0,\y,\z) (\x,0,\z)--(\x,0,0)--(\x,\y,0)--(0,\y,0)--(0,\y,\z) ;%

\draw[red,thick,->] (\x,\y/2,\z/2)--++(\radi/2,0,0) node[above]{$f$};
\draw[red,thick,->] (0,\y/2,\z/2)--++(-\radi/2,0,0) node[above]{$f$};

  \end{scope}

  \shade[ball color=red] ($\x*(-1.299cm,-0.75cm)+(\y/2,\z/2)$) circle (0.1);%
  \shade[ball color=red] ($0*(-1.299cm,-0.75cm)+(\y/2,\z/2)$) circle (0.1);%

\end{tikzpicture}
\end{document}
