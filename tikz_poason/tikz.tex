\documentclass[dvipdfmx]{jsarticle}
\usepackage{amsmath,amssymb}
%¥usepackage{mathpazo}
\usepackage{amsthm}
\newtheorem{dfn}{定義}
\newtheorem{thm}[dfn]{定理}
\newtheorem{lem}[dfn]{補題}
%\usepackage{newtxtext,newtxmath}
%\usepackage[hiresbb]{graphicx}
%¥図
\usepackage{bm}
%¥ボールド体
\usepackage{slashed}
\setlength{\textwidth}{\fullwidth}
\setlength{\textheight}{40\baselineskip}
\addtolength{\textheight}{\topskip}
\setlength{\voffset}{-0.2in}
\setlength{\topmargin}{-5pt}
\setlength{\headheight}{0pt}
\setlength{\headsep}{0pt}
\setlength{\textheight}{23.5cm}
\addtolength{\footskip}{3mm}
%\usepackage{mathrsfs}%ラグラジアン
\usepackage{wrapfig}%¥図の周りに文章を回り込ませる
\usepackage[dvipdfmx]{graphicx,color}
\usepackage{pgfplots}
\pgfplotsset{compat=1.16}
\usetikzlibrary{intersections,calc,arrows,decorations.pathmorphing,backgrounds,positioning,fit,petri}
\usepackage{array}
\usepackage{siunitx}
\usepackage[version=3]{mhchem}
\usepackage{float}
\newcommand{\setN}{\mathbb{N}}
\newcommand{\setZ}{\mathbb{Z}}
\newcommand{\setQ}{\mathbb{Q}}
\newcommand{\setR}{\mathbb{R}}
\newcommand{\setC}{\mathbb{C}}
\usepackage{physics}
\usepackage{tcolorbox}
\usepackage[dvipdfmx]{hyperref}
\usepackage{pxjahyper}
\usepackage{tikz}
%\usepackage{gnuplot-lua-tikz}
\usepackage{subfigure}
\hypersetup{colorlinks=true}
\makeatletter
\@addtoreset{equation}{section}
\def\theequation{\thesection.\arabic{equation}}
\makeatother
\begin{document}
\begin{tikzpicture}
[
x={(0.866cm,-0.5cm)}, y={(0.866cm,0.5cm)}, z={(0cm,1cm)},
scale=1.0,
>=latex,
inner sep=0pt, outer sep=2pt,
temple/.style={color=black,opacity=1},
axis/.style={thick, ->}
]
%\draw[help lines] (0,0) grid (10,10);

 % Frame
    \coordinate (O) at (0, 0, 0);
    \coordinate (a) at (-3, -4, 3);
    \coordinate (b) at (10, -4, 3);
    \coordinate (c) at (-3, -4, -3);
    \coordinate (d) at (10, -4, -3);
    \draw[axis] (O) -- +(14, 0,   0) node [right] {x};
    \draw[axis] (O) -- +(0,  2.5, 0) node [right] {y};
    \draw[axis] (O) -- +(0,  0,   2) node [above] {z};
	\draw[] (a)--(b);
	\draw[] (c)--(d);
	

    \draw[thick,dashed] (-2,0,0) -- (O);
\end{tikzpicture}
\end{document}
